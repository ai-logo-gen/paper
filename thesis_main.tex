%%%%%%%%%%%%%%%%%%%%%%%%%%%%%%%%%%%%%%%%%%%%%%%%%%%%%%%%%%%%%%%%%%%%%%%%%%%%
%% Springer Nature Journal Template - Math and Physical Sciences
%% Using sn-mathphys-num style (numbered references)
%%%%%%%%%%%%%%%%%%%%%%%%%%%%%%%%%%%%%%%%%%%%%%%%%%%%%%%%%%%%%%%%%%%%%%%%%%%%

\documentclass[pdflatex,sn-mathphys-num]{sn-jnl}

%%%% Standard Packages
\usepackage{graphicx}
\usepackage{multirow}
\usepackage{amsmath,amssymb,amsfonts}
\usepackage{amsthm}
\usepackage{mathrsfs}
\usepackage{booktabs}
\usepackage{xcolor}
\usepackage{listings}
\usepackage{subcaption}
\usepackage{algorithm}
\usepackage{algpseudocode}
\usepackage{float}

%%%% Code listing style
\definecolor{codegreen}{rgb}{0,0.6,0}
\definecolor{codegray}{rgb}{0.5,0.5,0.5}
\definecolor{codepurple}{rgb}{0.58,0,0.82}
\definecolor{backcolour}{rgb}{0.99,0.99,0.99}

\lstdefinestyle{mystyle}{
    backgroundcolor=\color{backcolour},   
    commentstyle=\color{codegreen},
    keywordstyle=\color{magenta},
    numberstyle=\tiny\color{codegray},
    stringstyle=\color{codepurple},
    basicstyle=\ttfamily\footnotesize,
    breakatwhitespace=false,         
    breaklines=true,                 
    captionpos=b,                    
    keepspaces=true,                 
    numbers=left,                    
    numbersep=5pt,                  
    showspaces=false,                
    showstringspaces=false,
    showtabs=false,                  
    tabsize=2
}

\lstset{style=mystyle}

%%%% Custom commands
\newcommand{\langde}[1]{}
\newcommand{\langen}[1]{#1}

%%%% Theorem environments (if needed)
\theoremstyle{thmstyleone}
\newtheorem{theorem}{Theorem}
\newtheorem{lemma}[theorem]{Lemma}
\newtheorem{conjecture}[theorem]{Conjecture}

\newcommand{\rr}{\mathbb{R}}
\newcommand{\al}{\alpha}
\DeclareMathOperator{\conv}{conv}
\DeclareMathOperator{\aff}{aff}

%%%%%%%%%%%%%%%%%%%%%%%%%%%%%%%%%%%%%%%%%%%%%%%%%%%%%%%%%%%%%%%%%%%%%%%%%%%%
%% Title and Author Information
%%%%%%%%%%%%%%%%%%%%%%%%%%%%%%%%%%%%%%%%%%%%%%%%%%%%%%%%%%%%%%%%%%%%%%%%%%%%

\title[Democratizing High-Quality Logo Design with Efficient AI]{The Minimalist Revolution: Democratizing High-Quality Logo Design with Efficient AI}

%%=============================================================%%
%% Author information - UPDATE WITH YOUR ACTUAL INFORMATION
%%=============================================================%%
\author*[1,2]{\fnm{Paul} \sur{Hornig}}\email{paul-hornig@gmx.de}

\affil[1]{\orgdiv{Data Scientist \& Software Engineer}, \orgname{BWI GmbH}, \orgaddress{\city{Magdeburg}, \country{Germany}}}

\affil[2]{\orgdiv{Big Data \& Business Analytics}, \orgname{FOM Hochschule für Oekonomie \& Management}, \orgaddress{\street{Leimkugelstraße 6}, \city{Essen}, \postcode{45141}, \country{Germany}}}

%%=============================================================%%
%% Abstract - moved from einleitung.tex
%%=============================================================%%
\abstract{This paper explores the resource-efficient fine-tuning of a Stable Diffusion model for the generation of minimalist logos. By leveraging a multimodal approach that combines text prompts with structural sketches via ControlNet, we address the limitations of standard text-to-image models in adhering to specific design constraints. We fine-tune the model on a curated dataset of 1,500 minimalist logos using consumer-grade hardware. A core contribution of this work is a systematic analysis of Low-Rank Adaptation (LoRA) hyperparameters, identifying optimal configurations for stylistic coherence and structural fidelity. We present a quantitative evaluation using CLIP, FID, and SSIM metrics, alongside qualitative case studies, demonstrating that high-quality, controllable design automation is achievable with limited computational resources.}

%%=============================================================%%
%% Keywords
%%=============================================================%%
\keywords{Generative AI, Logo Design, Stable Diffusion, LoRA, ControlNet, Parameter-Efficient Fine-Tuning}

%%=============================================================%%
%% Bibliography setup - sn-jnl handles this internally
%% Keep your .bib file reference
%%=============================================================%%
% Note: sn-jnl uses its own bibliography style
% The .bib file will be referenced at the end with \bibliography{}

\begin{document}

\maketitle

\section{Introduction}\label{sec:introduction}
Minimalist logo design relies on reduction, clarity, and structural precision - qualities that are often challenging for general purpose generative models to achieve consistently. While large-scale text-to-image models excel at artistic composition, they frequently struggle to produce the clean, vector-like aesthetics required for professional branding or to strictly adhere to a user's layout constraints \cite[1]{bertao2023blind}.

This work presents a specialized pipeline for generating minimalist logos by fine-tuning Stable Diffusion v1.5 \cite{HuggingFace_StableDiffusionv15_ModelCard}. To ensure both semantic relevance and structural control, we employ a hybrid conditioning strategy: text prompts define the style and content, while ControlNet \cite{ZHANG2023}\cite[8]{RAMESH2022}\cite[5]{NICHOL2021} enforces the geometric structure based on input sketches. Unlike approaches requiring massive datasets and industrial compute clusters, we focus on resource efficiency. We demonstrate that a compact, high-quality dataset of 1,500 examples is sufficient to adapt the model to the minimalist domain using consumer-grade hardware (NVIDIA RTX 5080).

Our research focuses on the optimization of the fine-tuning process itself. We conduct a rigorous analysis of Hyperparameters within the Low-Rank Adaptation (LoRA) technique, specifically examining the impact of learning rates and rank dimensions on the trade-off between training stability and generation quality.

The evaluation is primarily quantitative, utilizing the Fréchet Inception Distance (FID) to assess image quality, the CLIP score for semantic alignment, and the Structural Similarity Index (SSIM) to measure fidelity to the input sketches. We complement these metrics with qualitative case studies that illustrate the model's capability to translate rough sketches into polished, minimalist designs. This approach validates that accessible hardware and efficient training strategies can yield professional-grade design automation tools.

\subsection{Hypotheses}
Based on the defined objectives, we postulate the following hypotheses:
\begin{itemize}
	\item \textbf{Hypothesis 1 (H1):} Additional conditioning of the model via a sketch leads to a significantly higher structural correspondence with the design intent compared to purely text-conditioned results.
	\item \textbf{Hypothesis 2 (H2):} Optimizing an image generator through fine-tuning on a specialized logo dataset significantly improves the ability to generate stylistically coherent logos, as reflected in established image quality metrics.
	\item \textbf{Hypothesis 3 (H3):} Specific hyperparameter configurations have a direct and measurable influence on result quality, with an optimal configuration leading to better visual quality, structural correspondence, and text-image coherence.
	\item \textbf{Hypothesis 4 (H4):} A resource-efficiently optimized prototype generates results on consumer-grade hardware that exhibit higher quality in realizing design intent and greater commercial relevance in a qualitative evaluation compared to the unspecialized base model.
\end{itemize}

\section{Theoretical Background and Related Work}\label{sec:grundlagen}

\subsection{Minimalist Logo Design}\label{subsec:minimalistisches_logodesign}
\subsubsection*{Principles and Criteria}\label{subsubsec:prinzipien_und_kriterien}

The design philosophy of minimalism is scientifically grounded in Ockham's Razor, which posits that among functionally equivalent alternatives, the simplest is preferable \cite[172]{Lidwell2010}. In the context of logo design, minimalism aims to reduce a brand identity to its essential elements to achieve maximum \textbf{clarity}, recognizability, and functionality. A minimalist logo eliminates superfluous details and complex structures in favor of simple shapes, clear lines, and a limited color palette \cite[52]{Wheeler2017}. This reduction enhances \textbf{memorability}, as unique, simple forms are easier to store in visual memory \cite[2]{hjalmarsson2021impact}. Furthermore, it promotes \textbf{timelessness} by avoiding trends \cite[40]{Wheeler2017} and ensures \textbf{versatility}, allowing the design to remain recognizable across all scales and media \cite[44]{Wheeler2017}.

\subsubsection*{Topology of Minimalist Logos}\label{subsubsec:die_topologie_der_marken}
Following \citet[51]{Wheeler2017}, minimalist logos can be classified into three main categories:
\begin{itemize}
	\item \textbf{Typographic Logos:} Wordmarks (e.g., Google) or Monograms (e.g., IBM) that rely on type and negative space \cite[68]{Lupton2010}.
	\item \textbf{Pictorial Marks:} Symbols representing the brand. These include Emblems (text inside symbol), Pictorial Marks (literal representations like Apple), and Abstract Marks (geometric forms like Nike) \cite{rashgraphic2024}.
	\item \textbf{Combination Marks:} Integrating text and symbol (e.g., PayPal) to reinforce the brand message \cite{rashgraphic2024}.
\end{itemize}

\subsection{Generative AI Models}\label{subsec:generative_ac_ki-modelle}
Generative AI has evolved significantly from Generative Adversarial Networks (GANs) \cite{GOODFELLOW2014}, which often suffered from training instability \cite[2-3]{arjovsky2017wasserstein}. Denoising Diffusion Probabilistic Models (DDPMs) \cite[3]{HO2020} introduced a more stable paradigm based on reversing a noise addition process.
\citet{ROMBACH2022} further advanced this with Latent Diffusion Models (LDMs), which operate in a compressed latent space. This approach, used in Stable Diffusion, drastically reduces computational requirements, enabling high-resolution generation on consumer hardware.

\subsection{Conditioned Modeling}\label{subsec:konditionierte_modellierung}
To control generation, models combine modalities, typically text and image \cite{ZHANG2023}.

\subsubsection*{CLIP: Bridging Text and Image}\label{subsubsec:clip}
\citet{radford2021learning} introduced CLIP (Contrastive Language-Image Pre-training), which learns a joint embedding space for text and images. This allows diffusion models to be guided by text prompts, aligning the generated image with semantic descriptions \cite{RAMESH2022}.

\subsubsection*{ControlNet: Structural Control}\label{subsubsec:controlnet}
While CLIP provides semantic control, it lacks spatial precision. ControlNet \cite{ZHANG2023} addresses this by adding a trainable copy of the model's encoder blocks. This allows the injection of structural conditions, such as edge maps or sketches, directly into the generation process without retraining the base model. This is crucial for logo design, where specific shapes must be preserved.

\subsection{Parameter-Efficient Fine-Tuning (PEFT)}\label{subsec:finetuning}
Fine-tuning large foundation models is resource-intensive. PEFT methods aim to adapt models by training only a small subset of parameters \cite[2]{zhang2025parameter}.
\textbf{Low-Rank Adaptation (LoRA)} \cite[4]{HU2021} is a prominent PEFT technique. It hypothesizes that weight updates have a low intrinsic rank and approximates them using low-rank matrices ($ \Delta W = B \cdot A $). This significantly reduces the number of trainable parameters and memory usage, making fine-tuning feasible on consumer GPUs \cite[4]{Zhihao_2025}.

\subsection{Evaluation Metrics}\label{subsec:bewertungsmetriken}
We employ three metrics to evaluate the generated logos:
\begin{itemize}
	\item \textbf{CLIP Score:} Measures the semantic alignment between the generated image and the text prompt \cite{Hessel2021CLIPScoreAR}.
	\item \textbf{SSIM (Structural Similarity Index):} Quantifies the structural fidelity of the generated logo to the input sketch, focusing on luminance, contrast, and structure \cite{ssim_original}\cite[3]{snell2017learning}.
	\item \textbf{FID (Fréchet Inception Distance):} Assesses the realism and quality of generated images by measuring the distance between the feature distributions of real and generated images \cite[6]{heusel2017gans}.
\end{itemize}

\input{kapitel/methodik}
\input{kapitel/implementierung}
\newpage
\section{Results}\label{sec:results}

\subsection{Quantitative Evaluation}
This section presents the quantitative assessment of the model's performance based on the metrics CLIP, SSIM, and FID, as defined in Chapter \ref{sec:eval_pipe}. The evaluation follows a two-stage process tracked via MLflow: First, we analyze the model's sensitivity to LoRA rank ($r$), learning rate ($lr$), and target modules (``attn\_only'' vs. ``extended'') by evaluating 20 distinct hyperparameter combinations on the validation set. Subsequently, the optimal configuration identified from this analysis is validated on the unseen test set to confirm its generalization capabilities (see Section \ref{sec:testresults}).
\newpage
\subsubsection*{Loss Curve Analysis}
Validation loss serves as an indicator of generalization. Figures \ref{fig:val_loss_curves_attn_only} and \ref{fig:val_loss_curves_extended} illustrate the training dynamics.

\begin{figure}[hbt!]
    \centering
    \begin{subfigure}{\textwidth}
        \centering
        \includegraphics[width=\textwidth]{abbildungen/att_val_loss_en.png}
        \caption{Configuration ``attn\_only''}
        \label{fig:val_loss_curves_attn_only}
    \end{subfigure}
    \vfill
    \begin{subfigure}{\textwidth}
        \centering
        \includegraphics[width=\textwidth]{abbildungen/ext_val_loss_en.png}
        \caption{Configuration ``extended''}
        \label{fig:val_loss_curves_extended}
    \end{subfigure}
    \caption{Comparison of validation loss curves for ``attn\_only'' and ``extended'' configurations across hyperparameters.}
    \label{fig:val_loss_curves}
\end{figure}

\paragraph*{Observations for ``attn\_only''}
The learning rate is the primary driver, with $lr=1e-4$ yielding the lowest loss. The LoRA rank has a minor impact, mostly noticeable at $lr=1e-5$ where higher ranks perform slightly better. A notable exception is the combination of $r=4$ and $lr=1e-5$, which exhibits remarkably low variance, indicating a very stable learning process despite not achieving the absolute lowest loss.

\paragraph*{Observations for ``extended''}
The learning rate is even more dominant here, with significant gaps between $lr=1e-4$ and lower rates. While rank influence is generally low at the optimal learning rate, the combination of high rank ($r=32$) and lowest learning rate ($lr=1e-6$) shows an interesting anomaly: it achieves lower loss than the $lr=1e-5$ configuration towards the end of training, suggesting that lower learning rates might benefit from extended training durations.

\paragraph*{Synthesis}
The analysis reveals that the learning rate is the dominant factor. For both configurations, $lr=1e-4$ consistently yields the lowest loss. The ``attn\_only'' models generally achieve lower loss levels with less variance compared to ``extended'' models, which exhibit higher sensitivity and fluctuations. This instability is a known phenomenon in LoRA fine-tuning of diffusion models \cite{luo2024privacypreservinglowrankadaptationmembership}.

\subsubsection*{Structural Fidelity (SSIM)}
SSIM measures how well the generated logo adheres to the input sketch. Data analysis reveals a complex interaction between learning rate, LoRA rank, and target modules, rather than a single dominant factor. No single hyperparameter shows a consistently superior trend.

\begin{figure}[hbt!]
    \centering
    \includegraphics[width=\textwidth]{abbildungen/ssim_en.png}
    \caption{SSIM scores vs. learning rate, rank, and module configuration.}
    \label{fig:ssim_vs_rank_lr}
\end{figure}

As shown in Figure \ref{fig:ssim_vs_rank_lr}, the pre-trained base model (dotted line) already achieves an excellent SSIM of 0.819. Fine-tuning does not significantly improve structural fidelity; in fact, higher learning rates ($1e-4$) in the ``extended'' configuration can slightly degrade it. This confirms that ControlNet alone provides robust structural control.

\subsubsection*{Semantic Alignment (CLIP Score)}
The CLIP score evaluates the semantic correspondence between the image and the text prompt.

\begin{figure}[hbt!]
    \centering
    \includegraphics[width=\textwidth]{abbildungen/clip_en.png}
    \caption{CLIP scores vs. learning rate, rank, and module configuration.}
    \label{fig:clip_vs_rank_lr}
\end{figure}

The analysis reveals complex interactions between the hyperparameters. Although the absolute differences in CLIP scores are small and within a small percentage range, clear trends can be identified: In contrast to SSIM, fine-tuning improves semantic alignment (Figure \ref{fig:clip_vs_rank_lr}). The best performance is achieved with $lr=1e-4$, high rank ($r=32$), and the ``extended'' configuration, surpassing the base model by approximately 3.6\%. This highlights the necessity of fine-tuning for capturing domain-specific semantics.

\subsubsection*{Image Quality (FID)}
FID assesses the realism and feature distribution of the generated images.

\begin{figure}[hbt!]
    \centering
    \includegraphics[width=\textwidth]{abbildungen/fid_en.png}
    \caption{FID scores vs. learning rate, rank, and module configuration.}
    \label{fig:fid_vs_rank_lr}
\end{figure}

Figure \ref{fig:fid_vs_rank_lr} demonstrates that learning rate is the critical driver for image quality. The highest learning rate ($1e-4$) consistently produces the lowest (best) FID scores, improving upon the base model by roughly 31\%. The ``extended'' configuration outperforms ``attn\_only'' at this optimal learning rate.

\subsubsection*{Model Selection}
To identify the optimal model, a combined metric normalizing SSIM, CLIP, and (inverted) FID was calculated.

\begin{figure}[hbt!]
    \centering
    \includegraphics[width=\textwidth]{abbildungen/best_models.png}
    \caption{Top 5 model configurations (normalized metrics).}
    \label{fig:best_models}
\end{figure}

The radar chart in Figure \ref{fig:best_models} identifies the configuration with \textbf{Rank 32, $lr=1e-4$, and ``extended'' modules} as the best overall performer. It offers the best compromise, maximizing image quality and semantic alignment while maintaining acceptable structural fidelity.

\subsubsection*{Evaluation on Test Set}\label{sec:testresults}
The selected model was evaluated on an unseen test set to assess generalization. To ensure statistical robustness, the evaluation was conducted using five different random seeds. Table \ref{tab:testresults} compares the base model (without and with ControlNet) against the fine-tuned model.

\begin{table}[hbt!]
    \caption{Test set metrics comparison (mean $\pm$ standard deviation over 5 seeds).}
    \label{tab:testresults}
    \centering
    \begin{tabular}{lccc}
        \toprule
        \textbf{Model}                         & \textbf{CLIP Score} $\uparrow$ & \textbf{FID} $\downarrow$ & \textbf{SSIM} $\uparrow$ \\
        \midrule
        Base Model (w/o CN)                    & $0.285 \pm 0.001$              & $223.3 \pm 4.1$           & $0.621 \pm 0.010$        \\
        Base Model (w/ CN)                     & $0.277 \pm 0.002$              & $228.3 \pm 5.5$           & $0.819 \pm 0.011$        \\
        Finetuned Model                        & $0.287 \pm 0.002$              & $157.6 \pm 1.2$           & $0.788 \pm 0.004$        \\
        \midrule
        Improvement (Finetuned vs. Base w/ CN) & +3.6\%                         & -31.0\%                   & -3.8\%                   \\
        \bottomrule
    \end{tabular}
\end{table}

The results demonstrate the distinct roles of the components:
\begin{itemize}
    \item \textbf{ControlNet} is essential for structure, boosting SSIM by 31.9\% compared to the unconditioned base model.
    \item \textbf{Fine-tuning} enhances semantic alignment (+3.6\% CLIP) and drastically improves image quality (-31.0\% FID). The significant FID reduction ($228.3$ to $157.6$) outweighs the minor 3.8\% SSIM decrease ($0.819$ to $0.788$), which is negligible as the structural fidelity remains at an excellently high level.
\end{itemize}

\subsection{Qualitative Analysis}\label{sec:qualitative_analyse}
To validate the quantitative findings, a qualitative analysis was conducted using five representative case examples (E1--E5), as illustrated in Figure \ref{fig:logo_case_studies}. Four examples (E1, E2, E3, E5) were selected based on their sketches from the test dataset to cover diverse logo types: wordmarks, pictorial/abstract marks, and combined marks. Additionally, Example E4 - a monogram based on a hand-drawn sketch - was included to test the model on authentic human input. This diversity allows for a comprehensive evaluation across design categories.

For each case, both the base model (Stable Diffusion v1.5 with ControlNet) and the fine-tuned model generated a logo using the corresponding sketch and text prompt. To ensure objectivity and reproducibility, generation was performed with exactly one iteration per model, without any post-selection or optimization (no cherry-picking).

\noindent\textbf{Negative Prompt (for all case examples):} \textit{sketch, photorealistic, pattern in background, noisy, blurry, watermark}

\begin{figure}[hbt!]
    \centering
    \includegraphics[width=\textwidth]{abbildungen/comparison_overview.png}
    \caption{Comparison of generated logos across five representative cases (E1-E5) between the base and fine-tuned models.}
    \label{fig:logo_case_studies}
\end{figure}
\subsubsection*{Stylistic Evaluation}
The fine-tuned model demonstrates a superior ability to implement specific design constraints compared to the base model.
\begin{itemize}
    \item \textbf{Strengths:} It reliably generates ``solid backgrounds'' and consistent, minimalist color palettes, avoiding the unwanted textures often produced by the base model.
    \item \textbf{Weaknesses:} Challenges remain in fine-grained details. For instance, text rendering (Case E3) can be deformed, and specific gradient instructions (Case E4) are occasionally ignored.
\end{itemize}
Despite these limitations, the fine-tuned model shows a significant improvement in generating commercially viable, minimalist logos.

\newpage
\section{Discussion}\label{sec:diskussion}

This study systematically investigated the resource-efficient specialization of a multimodal diffusion model for minimalist logo generation. By combining LoRA-based fine-tuning with ControlNet guidance, a prototype was developed that effectively processes both textual and structural constraints. This chapter interprets the experimental findings by directly addressing the research hypotheses formulated in Chapter \ref{sec:introduction} and discusses limitations alongside future research directions.

\subsection{Interpretation of Results and Hypothesis Confirmation}

The experimental results demonstrate the efficacy of the proposed hybrid approach in achieving professional-grade design automation with limited resources. This section synthesizes the findings to confirm the research hypotheses.

\subsubsection*{Efficacy of the Hybrid Strategy (H1 \& H2)}
The combination of ControlNet and LoRA fine-tuning proved highly effective, confirming the fundamental hypotheses regarding structure and quality.
\textbf{Hypothesis H1}, stating that sketch conditioning leads to significantly higher structural correspondence without loss of image quality, was clearly confirmed. Evaluation showed a dramatic increase in the SSIM score from 0.621 for the base model without ControlNet to 0.819 with ControlNet (+31.9\%), while maintaining overall aesthetic markers. Although the FID score on the test set showed a slight fluctuation from 223.3 to 228.3 (+2.2\%), this marginal change is considered non-significant in the context of the substantial structural gain, thereby supporting the hypothesis. This validates the theoretical considerations of \citet{ZHANG2023}, positioning ControlNet as a solution for the geometric limitations of pure text-to-image models.

Simultaneously, \textbf{Hypothesis H2}, postulating quality improvement through domain-specific fine-tuning, was fully confirmed. The best fine-tuned model achieved a CLIP score of 0.287 vs. 0.277 for the base model (+3.6\%) and an FID reduction from 228.3 to 157.6 (-31.0\%). These results align with \citet{ruiz2023dreamboothfinetuningtexttoimage}, validating significant gains even with limited iterations. The efficiency of this combination is particularly noteworthy, as significant quality improvements were achieved with a compact dataset and limited resources. Notably, the slight SSIM drop (0.819 to 0.788) reflects a desired tradeoff: shifting focus from rigid structure to semantic accuracy and visual quality suitable for minimalist logos.

\subsubsection*{Hyperparameter Dynamics (H3)}
The systematic analysis confirmed \textbf{Hypothesis H3}, identifying the learning rate as the dominant factor for both semantic coherence (CLIP) and visual quality (FID). A high learning rate of $1e-4$ combined with the ``extended'' configuration yielded the best overall results. Higher LoRA ranks (32) tended to enable better semantic coherence at high learning rates, consistent with \citet{HU2021}, suggesting that higher ranks approach full fine-tuning capacity. The findings highlight the importance of careful hyperparameter tuning for LoRA, where a high learning rate combined with extended module adaptation provides the best balance between plasticity and stability. However, it is essential to emphasize that these observations are specific to our experimental design and should not be generalized without further validation on different architectures or datasets.

\subsubsection*{Qualitative Assessment and Practical Viability (H4)}
\textbf{Hypothesis H4}, suggesting the superiority of the optimized prototype in qualitative assessment, was supported by the stylistic analysis. The evaluation (see Section \ref{sec:qualitative_analyse}) indicated that the fine-tuned model better adheres to the minimalist design intent compared to the base model. Specifically, it demonstrated improved capabilities in generating solid backgrounds and harmonic color palettes suitable for professional branding. While the base model frequently introduced unwanted textures, the optimized prototype produced cleaner, vector-like aesthetics. This underscores the model's practical potential for bridging the gap between rough ideation and polished design, effectively handling authentic human inputs despite remaining challenges in fine-grained detail, such as precise text rendering and geometric sharpness.


\subsection{Limitations and Future Work}

While the results demonstrate the viability of the proposed pipeline, several limitations identify key areas for future optimization. A central challenge remains the consistent generation of perfect fine-grained details, such as razor-sharp geometric lines (e.g., in monograms) and highly legible typography. This constraint is primarily attributed to four factors:
\begin{enumerate}
    \item \textbf{Data Quantity:} The specialized training subset was relatively small (1,500 images) to accommodate iterative hyperparameter optimization.
    \item \textbf{Data Quality:} The semi-structured captions from the source dataset often lacked the descriptive depth required for the model to learn complex stylistic and geometric nuances.
    \item \textbf{Training Duration:} To maintain resource efficiency on consumer hardware, the number of training steps was kept relatively low, which may have limited the convergence of fine-grained structural features.
    \item \textbf{Synthetic Artifacts:} The generation of synthetic training data sometimes introduced unwanted lines or thematic elements (e.g., pencils or rough textures) not present in the original logo input. These artifacts occurred because the model associated the "sketch" domain with physical drawing tools, potentially polluting the structural control signal.
\end{enumerate}

Despite these limitations, the proposed method is recommended for productive use in the early stages of the design process, specifically for ideation and the rapid generation of stylized layout concepts. Future work will focus on the following pillars:

\subsubsection*{Data Quality and Augmentation}
Future iterations should employ Vision-Language Models (VLMs) to generate ultra-detailed captions and structured categorizations, such as distinguishing wordmarks from pictorial marks. This would significantly enhance the model's semantic understanding \cite{zeng-etal-2025-enhancing-large}. Furthermore, synthetic sketch generation could be diversified using more advanced edge detection algorithms or dedicated sketch models to improve robustness against the variability of human hand-drawn inputs. Crucially, future data pipelines should prioritize the logical traceability of synthetic signals, ensuring that generated sketches are strictly consistent with the original logo templates and free from non-semantic artifacts that do not reflect the underlying design.

\subsubsection*{Modeling Efficiency and Stability}
Implementation of QLoRA (4-bit quantization) \cite{dettmers2023qloraefficientfinetuningquantized} and ``torch.compile()'' would allow for larger batch sizes and faster training on consumer hardware. Furthermore, Bayesian optimization could replace grid search to more efficiently explore the hyperparameter space and identify regions of stable convergence, addressing the training instability sometimes observed in LoRA fine-tuning \cite{luo2024privacypreservinglowrankadaptationmembership}.

\subsubsection*{Evaluation Metrics}
As standard metrics like CLIP, FID, and SSIM do not fully capture design-specific criteria such as memorability or vectorizability, future work should focus on developing domain-specific evaluation frameworks. This includes automated vectorization success rates and larger-scale user studies with professional designers to provide more robust and practically relevant quality assessments.

\section{Conclusion and Outlook}\label{sec:fazit}

\subsection{Summary}
This thesis investigated the resource-efficient optimization of a multimodal diffusion model for minimalist logo generation. By combining LoRA-based fine-tuning with ControlNet guidance, we developed a prototype capable of operating on consumer-grade hardware (NVIDIA RTX 5080).
The experimental evaluation confirmed that structural guidance is indispensable for geometric precision (SSIM +32.4\%), while domain-specific fine-tuning significantly enhances semantic and visual quality (FID -30.9\%). Systematic analysis identified the learning rate as the critical lever for convergence within the scope of this work, with higher rates ($1e-4$) consistently yielding superior results.

\subsection{Scientific Contributions and Implications}
A key contribution of this work is the validation of parameter-efficient training strategies for the specific domain of logo design.
\begin{itemize}
    \item \textbf{Hyperparameter Dynamics:} Our findings confirm literature recommendations for high learning rates ($1e-4$) in PEFT methods \cite{HU2021}. However, contrary to the suggestion that very low ranks (4 or 8) constitute a "sweet spot" \cite{HU2021}, our experiments demonstrated that a higher rank of 32 provided the necessary capacity to capture the stylistic nuances of minimalist design without overfitting.
    \item \textbf{Efficiency Verification:} We demonstrated that a compact, well-filtered dataset of 1,500 high-quality logos and approximately 2,500 training steps are sufficient to achieve professional-grade results. This underscores that data quality and curation significantly outweigh pure volume in effective domain adaptation.
\end{itemize}

\subsection{Future Directions}
Future optimization should prioritize data quality over quantity. The integration of Vision-Language Models (VLMs) offers a promising avenue to replace simplistic tag-based captions with dense, descriptive natural language, significantly improving semantic alignment. Furthermore, advances in synthetic data generation and evolving model architectures suggest that the barrier to entry for training specialized, high-quality generative models will continue to decrease.
Generative AI establishes itself not as a replacement for human expertise, but as a powerful instrument for exploration and iteration, democratizing access to professional design tools.


%%=============================================================%%
%% Acknowledgments (Optional) - Remove if not needed
%%=============================================================%%
% %%%%%%%%%%%%%%%%%%%%%%%%%%%%%%%%%%%%%%%%%%%%%%%%%%%%%%%%%%%%%%%%%%%%%%%%%%%%
%% Acknowledgments Section (Optional)
%% This section should appear before the Declarations section
%% UPDATE OR REMOVE THIS SECTION AS NEEDED
%%%%%%%%%%%%%%%%%%%%%%%%%%%%%%%%%%%%%%%%%%%%%%%%%%%%%%%%%%%%%%%%%%%%%%%%%%%%

\section*{Acknowledgments}

% Example acknowledgments - UPDATE WITH YOUR ACTUAL ACKNOWLEDGMENTS OR REMOVE THIS FILE

The author would like to thank [names] for their valuable feedback and support during this research. We acknowledge the use of computational resources provided by [institution/organization if applicable]. We also thank the creators of the \texttt{iamkaikai/amazing\_logos\_v4} dataset for making their data publicly available.

% If you have no acknowledgments, you can delete this file and remove the %%%%%%%%%%%%%%%%%%%%%%%%%%%%%%%%%%%%%%%%%%%%%%%%%%%%%%%%%%%%%%%%%%%%%%%%%%%%
%% Acknowledgments Section (Optional)
%% This section should appear before the Declarations section
%% UPDATE OR REMOVE THIS SECTION AS NEEDED
%%%%%%%%%%%%%%%%%%%%%%%%%%%%%%%%%%%%%%%%%%%%%%%%%%%%%%%%%%%%%%%%%%%%%%%%%%%%

\section*{Acknowledgments}

% Example acknowledgments - UPDATE WITH YOUR ACTUAL ACKNOWLEDGMENTS OR REMOVE THIS FILE

The author would like to thank [names] for their valuable feedback and support during this research. We acknowledge the use of computational resources provided by [institution/organization if applicable]. We also thank the creators of the \texttt{iamkaikai/amazing\_logos\_v4} dataset for making their data publicly available.

% If you have no acknowledgments, you can delete this file and remove the %%%%%%%%%%%%%%%%%%%%%%%%%%%%%%%%%%%%%%%%%%%%%%%%%%%%%%%%%%%%%%%%%%%%%%%%%%%%
%% Acknowledgments Section (Optional)
%% This section should appear before the Declarations section
%% UPDATE OR REMOVE THIS SECTION AS NEEDED
%%%%%%%%%%%%%%%%%%%%%%%%%%%%%%%%%%%%%%%%%%%%%%%%%%%%%%%%%%%%%%%%%%%%%%%%%%%%

\section*{Acknowledgments}

% Example acknowledgments - UPDATE WITH YOUR ACTUAL ACKNOWLEDGMENTS OR REMOVE THIS FILE

The author would like to thank [names] for their valuable feedback and support during this research. We acknowledge the use of computational resources provided by [institution/organization if applicable]. We also thank the creators of the \texttt{iamkaikai/amazing\_logos\_v4} dataset for making their data publicly available.

% If you have no acknowledgments, you can delete this file and remove the \input{kapitel/acknowledgments.tex} line from thesis_main.tex
 line from thesis_main.tex
 line from thesis_main.tex


%%=============================================================%%
%% Declarations section - REQUIRED by journal
%%=============================================================%%
%%%%%%%%%%%%%%%%%%%%%%%%%%%%%%%%%%%%%%%%%%%%%%%%%%%%%%%%%%%%%%%%%%%%%%%%%%%%
%% Declarations Section - REQUIRED by Springer Nature Journals
%% This section must appear before the references
%%%%%%%%%%%%%%%%%%%%%%%%%%%%%%%%%%%%%%%%%%%%%%%%%%%%%%%%%%%%%%%%%%%%%%%%%%%%

\section*{Declarations}

%%%%%%%%%%%%%%%%%%%%%%%%%%%%%%%%%%%%%%%%%%%%%%%%%%%%%%%%%%%%%%%%%%%%%%%%%%%%
%% Funding
%% UPDATE THIS SECTION WITH YOUR ACTUAL FUNDING INFORMATION
%%%%%%%%%%%%%%%%%%%%%%%%%%%%%%%%%%%%%%%%%%%%%%%%%%%%%%%%%%%%%%%%%%%%%%%%%%%%
\subsection*{Funding}
No funding was received for conducting this study.

% Alternative if you received funding:
% This study was funded by [funding organization] (Grant number [XXX]).

%%%%%%%%%%%%%%%%%%%%%%%%%%%%%%%%%%%%%%%%%%%%%%%%%%%%%%%%%%%%%%%%%%%%%%%%%%%%
%% Competing Interests
%% UPDATE THIS SECTION WITH YOUR ACTUAL COMPETING INTERESTS
%%%%%%%%%%%%%%%%%%%%%%%%%%%%%%%%%%%%%%%%%%%%%%%%%%%%%%%%%%%%%%%%%%%%%%%%%%%%
\subsection*{Competing Interests}
The authors have no competing interests to declare that are relevant to the content of this article.

% Alternative if you have competing interests:
% Financial interests: [Describe any financial interests]
% Non-financial interests: [Describe any non-financial interests]

%%%%%%%%%%%%%%%%%%%%%%%%%%%%%%%%%%%%%%%%%%%%%%%%%%%%%%%%%%%%%%%%%%%%%%%%%%%%
%% Data Availability
%% REQUIRED for original research articles
%% UPDATE THIS SECTION WITH YOUR ACTUAL DATA AVAILABILITY INFORMATION
%%%%%%%%%%%%%%%%%%%%%%%%%%%%%%%%%%%%%%%%%%%%%%%%%%%%%%%%%%%%%%%%%%%%%%%%%%%%
\subsection*{Data Availability}
The dataset used in this study is publicly available at Hugging Face: \texttt{iamkaikai/amazing\_logos\_v4} (\url{https://huggingface.co/datasets/iamkaikai/amazing_logos_v4}). The curated subset of 1,810 minimalist logos, along with the generated sketches and lineart maps, can be made available upon reasonable request to the corresponding author. The evaluation metrics and model checkpoints from the hyperparameter experiments are tracked via MLflow and can be shared for reproducibility purposes.

%%%%%%%%%%%%%%%%%%%%%%%%%%%%%%%%%%%%%%%%%%%%%%%%%%%%%%%%%%%%%%%%%%%%%%%%%%%%
%% Code Availability
%% Highly relevant for this computational research
%% UPDATE THIS SECTION WITH YOUR ACTUAL CODE REPOSITORY INFORMATION
%%%%%%%%%%%%%%%%%%%%%%%%%%%%%%%%%%%%%%%%%%%%%%%%%%%%%%%%%%%%%%%%%%%%%%%%%%%%
\subsection*{Code Availability}
The code for data preprocessing, model fine-tuning, and evaluation is available at [REPOSITORY URL - TO BE ADDED]. The implementation is based on the Hugging Face Diffusers library and uses standard open-source tools including Stable Diffusion v1.5, ControlNet, and LoRA adapters.

% Alternative if code is not yet public:
% The code used in this study is available from the corresponding author upon reasonable request.

%%%%%%%%%%%%%%%%%%%%%%%%%%%%%%%%%%%%%%%%%%%%%%%%%%%%%%%%%%%%%%%%%%%%%%%%%%%%
%% Author Contributions
%% UPDATE THIS SECTION WITH ACTUAL AUTHOR CONTRIBUTIONS
%%%%%%%%%%%%%%%%%%%%%%%%%%%%%%%%%%%%%%%%%%%%%%%%%%%%%%%%%%%%%%%%%%%%%%%%%%%%
\subsection*{Author Contributions}
All authors contributed to the study conception and design. Material preparation, data collection and analysis were performed by Paul Hornig. The first draft of the manuscript was written by Paul Hornig and all authors commented on previous versions of the manuscript. All authors read and approved the final manuscript.

% For single author:
% Paul Hornig is the sole author of this work and was responsible for all aspects of the research, including study conception and design, data collection and analysis, and manuscript preparation.

%%%%%%%%%%%%%%%%%%%%%%%%%%%%%%%%%%%%%%%%%%%%%%%%%%%%%%%%%%%%%%%%%%%%%%%%%%%%
%% Ethics Approval (if applicable)
%% Uncomment and update if your research involved human subjects or animals
%%%%%%%%%%%%%%%%%%%%%%%%%%%%%%%%%%%%%%%%%%%%%%%%%%%%%%%%%%%%%%%%%%%%%%%%%%%%
% \subsection*{Ethics Approval}
% Not applicable. This study did not involve human participants or animals.

%%%%%%%%%%%%%%%%%%%%%%%%%%%%%%%%%%%%%%%%%%%%%%%%%%%%%%%%%%%%%%%%%%%%%%%%%%%%
%% Consent (if applicable)
%% Uncomment and update if your research involved human subjects
%%%%%%%%%%%%%%%%%%%%%%%%%%%%%%%%%%%%%%%%%%%%%%%%%%%%%%%%%%%%%%%%%%%%%%%%%%%%
% \subsection*{Consent}
% Not applicable. This study did not involve human participants.


\newpage
\appendix
\section{Appendix}

\subsection{Evaluated Hyperparameter Combinations for LoRA Training}\label{app:hyperpara_kombis}
\begin{lstlisting}[language=Python, caption={Python code defining the hyperparameter combinations for LoRA training}, label={lst:traing_hyperparams}]
HYPERPARAMETER_MAP = {
    # Attention-only modules
    "attn_rank4_lr1e-5":  {"lora_r": 4, "learning_rate": 1e-5, "lora_target_modules": "attn_only"},
    "attn_rank4_lr1e-4":  {"lora_r": 4, "learning_rate": 1e-4, "lora_target_modules": "attn_only"},
    "attn_rank8_lr1e-5":  {"lora_r": 8, "learning_rate": 1e-5, "lora_target_modules": "attn_only"},
    "attn_rank8_lr1e-4":  {"lora_r": 8, "learning_rate": 1e-4, "lora_target_modules": "attn_only"},
    "attn_rank16_lr1e-6": {"lora_r": 16, "learning_rate": 1e-6, "lora_target_modules": "attn_only"},
    "attn_rank16_lr1e-5": {"lora_r": 16, "learning_rate": 1e-5, "lora_target_modules": "attn_only"},
    "attn_rank16_lr1e-4": {"lora_r": 16, "learning_rate": 1e-4, "lora_target_modules": "attn_only"},
    "attn_rank32_lr1e-6": {"lora_r": 32, "learning_rate": 1e-6, "lora_target_modules": "attn_only"},
    "attn_rank32_lr1e-5": {"lora_r": 32, "learning_rate": 1e-5, "lora_target_modules": "attn_only"},
    "attn_rank32_lr1e-4": {"lora_r": 32, "learning_rate": 1e-4, "lora_target_modules": "attn_only"},

    # Extended modules (Attention + MLP)
    "ext_rank4_lr1e-5":   {"lora_r": 4, "learning_rate": 1e-5, "lora_target_modules": "extended"},
    "ext_rank4_lr1e-4":   {"lora_r": 4, "learning_rate": 1e-4, "lora_target_modules": "extended"},
    "ext_rank8_lr1e-5":   {"lora_r": 8, "learning_rate": 1e-5, "lora_target_modules": "extended"},
    "ext_rank8_lr1e-4":   {"lora_r": 8, "learning_rate": 1e-4, "lora_target_modules": "extended"},
    "ext_rank16_lr1e-6":  {"lora_r": 16, "learning_rate": 1e-6, "lora_target_modules": "extended"},
    "ext_rank16_lr1e-5":  {"lora_r": 16, "learning_rate": 1e-5, "lora_target_modules": "extended"},
    "ext_rank16_lr1e-4":  {"lora_r": 16, "learning_rate": 1e-4, "lora_target_modules": "extended"},
    "ext_rank32_lr1e-6":  {"lora_r": 32, "learning_rate": 1e-6, "lora_target_modules": "extended"},
    "ext_rank32_lr1e-5":  {"lora_r": 32, "learning_rate": 1e-5, "lora_target_modules": "extended"},
    "ext_rank32_lr1e-4":  {"lora_r": 32, "learning_rate": 1e-4, "lora_target_modules": "extended"},
}

LORA_TARGET_MODULES_MAP = {
    "attn_only": ["to_q", "to_k", "to_v"],
    "extended": ["to_q", "to_k", "to_v", "to_out.0", "proj_in", "proj_out"],
}
\end{lstlisting}


%%=============================================================%%
%% Bibliography
%%=============================================================%%
\bibliography{literatur/literatur_ph}

\end{document}